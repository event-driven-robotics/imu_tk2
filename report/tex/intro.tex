\section{Introduction}\label{sec:intro}

The document summarizes the usage of the IMU calibration tool and all extra option developed in the IIT context.
%
The objective of this report is to introduce the IMU calibration tool, explain its options, briefly describe the work done in our context, and to log the state and partial results and conclusions.

The original work \cite{2014:Tedaldi} presents an IMU calibration tool which does not depend on specific equipment, not precise positioning of the IMU, which is compliant with our IMU system.
%
As such, in order to compute the calibration parameters, an optimization must be performed using several different IMU static positions.

Other methods use specific equipment to reduce the number of positions required for the calibration \cite{lv2016method}, to analytically calculate the calibration parameters \cite{bhatia2016development}. \cite{qureshi2017algorithm} proposes the same approach.

 




